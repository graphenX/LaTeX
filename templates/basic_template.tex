\documentclass[10pt, a4paper]{article} % basic document settings
%----------------------------------------------------------------------------------------
%	PACKAGES AND OTHER DOCUMENT CONFIGURATIONS
%----------------------------------------------------------------------------------------


\usepackage{microtype} % Better typography

\usepackage[svgnames]{xcolor} % Enabling colors by their 'svgnames'

\usepackage[hang, small, labelfont=bf, up, textfont=it]{caption} % Custom captions under/above tables and figures

\usepackage{booktabs} % Horizontal rules in tables

\usepackage{lastpage} % Used to determine the number of pages in the document (for "Page X of Total")

\usepackage{graphicx} % Required for adding images

\usepackage{enumitem} % Required for customising lists
\setlist{noitemsep} % Remove spacing between bullet/numbered list elements

\usepackage{sectsty} % Enables custom section titles
\allsectionsfont{\usefont{OT1}{phv}{b}{n}} % Change the font of all section commands (Helvetica)


%----------------------------------------------------------------------------------------
%	FONTS
%----------------------------------------------------------------------------------------

\usepackage[spanish]{babel}
\usepackage[utf8]{inputenc} % Required for inputting international characters
\usepackage[T1]{fontenc} % Output font encoding for international characters
\usepackage{XCharter} % Use the XCharter font

%----------------------------------------------------------------------------------------
%	MARGINS AND SPACING
%----------------------------------------------------------------------------------------

\usepackage{geometry} % Required for adjusting page dimensions

\geometry{
	top=1cm, % Top margin
	bottom=1.5cm, % Bottom margin
	left=2cm, % Left margin
	right=2cm, % Right margin
	includehead, % Include space for a header
	includefoot, % Include space for a footer
	% showframe, % Uncomment to show how the type block is set on the page
}


%----------------------------------------------------------------------------------------
%	FONTS
%----------------------------------------------------------------------------------------

\usepackage[T1]{fontenc} % Output font encoding for international characters
\usepackage[utf8]{inputenc} % Required for inputting international characters
\usepackage{XCharter} % Use the XCharter font

%----------------------------------------------------------------------------------------
%	HEADERS AND FOOTERS
%----------------------------------------------------------------------------------------

\usepackage{fancyhdr} % Needed to define custom headers/footers
\pagestyle{fancy} % Enables the custom headers/footers
\fancypagestyle{plain}%{
%\fancyfoot[LE, RO]{Apurba Paul}
%\fancyfoot[LO, RE]{Using fancyhdr}}


\renewcommand{\headrulewidth}{0.0pt} % No header rule
\renewcommand{\footrulewidth}{0.4pt} % Thin footer rule

\renewcommand{\sectionmark}[1]{\markboth{#1}{}} % Removes the section number from the header when \leftmark is used

%\nouppercase\leftmark % Add this to one of the lines below if you want a section title in the header/footer

%% Headers
\lhead{} % Left header
\chead{}
\rhead{} % Right header
% \chead{\textit{\thetitle}} % Center header - currently printing the article title


% Footers
\lfoot{} % Left footer
\cfoot{} % Center footer
\rfoot{\footnotesize Page \thepage\ of \pageref{LastPage}} % Right footer, "Page 1 of 2"

%----------------------------------------------------------------------------------------
%	TITLE SECTION
%----------------------------------------------------------------------------------------

\newcommand{\authorstyle}[1]{{\large\usefont{OT1}{phv}{b}{n}\color{DarkRed}#1}} % Authors style (Helvetica)

\newcommand{\institution}[1]{{\footnotesize\usefont{OT1}{phv}{m}{sl}\color{Black}#1}} % Institutions style (Helvetica)

\usepackage{titling} % Allows custom title configuration

\usepackage{titlesec} % Set title color

\titleformat{\section}
{\color{DarkRed}\normalfont\huge\bfseries}
{\color{DarkRed}\thesection}{1em}{}

\titleformat{\subsection}
{\color{DarkRed}\normalfont\LARGE\bfseries}
{\color{DarkRed}\thesubsection}{1em}{}

\titleformat{\subsubsection}
{\color{DarkRed}\normalfont\Large\bfseries}
{\color{DarkRed}\thesubsubsection}{1em}{}

\usepackage{xcolor} % Package tdo madify table of contencts color

\AtBeginDocument{ % to modify table of contents text size
  \addtocontents{toc}{\Large}
  \addtocontents{lof}{\Large}
}

%----------------------------------------------------------------------------------------
%	WATERMARK
%----------------------------------------------------------------------------------------

% \usepackage{draftwatermark}
% \SetWatermarkText{\includegraphics{watermark.png}}
% \SetWatermarkScale{0.5}

\usepackage{background}
\backgroundsetup{contents=\includegraphics{watermark.png}, scale=1, opacity=0.25, angle=0}




 % Specifies the document structure and loads requires packages

\usepackage{tabularx} % package for formating tables

\usepackage{blindtext} % package for random text on templates

%----------------------------------------------------------------------------------------
%	TITLE
%----------------------------------------------------------------------------------------

\title{\begin{flushleft}\color{DarkRed} \textbf{\huge{NGINX}}\end{flushleft}} % The article title title}{{\color{DarkRed}}

\author{
	\authorstyle{Daniel Fariña Sande\textsuperscript{1}} \\ \\
	\textsuperscript{1}\institution{Administrador de redes y sitemas. Especialista en ciberseguridad en Tecdesoft.} % Institution 1
}
%----------------------------------------------------------------------------------------
%	DOCUMENT START
%----------------------------------------------------------------------------------------
\begin{document}
    
\date{}
\maketitle

\begin{figure}[h]   % set image format
    \centering
    \includegraphics[width=1\textwidth] % introduce image on document
    {nginx}
    \caption{NGINX logo} % image name
    \label{fig:nginx1} % image label
\end{figure}

\newpage % ends content on current page

\begin{table}[h!] % this is how tables with lines are introduced
    \begin{center} 
        \begin{tabular}{|p{8cm}|p{3cm}|} 
        \hline
        EMISOR & FECHA \\ 
        \hline
        cell4 & cell5 \\
        \hline  
        cell7 & cell8 \\ 
        \hline 
        cell7 & cell8 \\ 
        \hline 
        cell7 & cell8 \\ 
        \hline 
        cell7 & cell8 \\ 
        \hline 
        cell7 & cell8 \\ 
        \hline 
        cell7 & cell8 \\ 
        \hline 
        cell7 & cell8 \\ 
        \hline 
        cell7 & cell8 \\ 
        \hline 
        cell7 & cell8 \\ 
        \hline 
        cell7 & cell8 \\ 
        \hline 
        cell7 & cell8 \\ 
        \hline 
        cell7 & cell8 \\ 
        \hline 
        \end{tabular}
    \end{center}
    \caption{Table to test captions and labels}
    \label{table:data1}
\end{table}

\begin{tabularx}{0.8\textwidth} { 
    | >{\raggedright\arraybackslash}X 
    | >{\centering\arraybackslash}X | }
    \hline
    EMISOR & FECHA \\ 
    \hline
    cell4 & cell5 \\
    \hline  
    cell7 & cell8 \\ 
    \hline 
    cell7 & cell8 \\ 
    \hline 
    cell7 & cell8 \\ 
    \hline 
    cell7 & cell8 \\ 
    \hline 
    cell7 & cell8 \\ 
    \hline 
    cell7 & cell8 \\ 
    \hline 
    cell7 & cell8 \\ 
    \hline 
    cell7 & cell8 \\ 
    \hline 
    cell7 & cell8 \\ 
    \hline 
    cell7 & cell8 \\ 
    \hline 
    cell7 & cell8 \\ 
    \hline 
    cell7 & cell8 \\ 
    \hline 
\end{tabularx}

\newpage % ends content on current page

\begin{abstract}
    On this section we explain than this document has been created to develop LaTeX skills and document
    nginx configurations.On this section we explain than this document has been created to develop LaTeX skills and document
    nginx configurations.On this section we explain than this document has been created to develop LaTeX skills and document
    nginx configurations.
\end{abstract}

\newpage % ends content on current page

\begingroup % start a TeX group
\color{DarkRed}% or whatever color you wish to use
\tableofcontents
\endgroup   % end of TeX group

\newpage % ends content on current page


\section{Introduction} % this is how you create sections and subsections, numbered and unnumbered

Hello world. This is the first section.

Lorem ipsum dolor sit amet, consectetuer adipiscing elit.

\section{Introduction}

Hello world. This is the first section.

Nginx logo looks like figure \ref{fig:nginx1}, but if we want to reference this form another page.Nginx logo looks like figure \ref{fig:nginx1}, but if we want to reference this form another page.
Nginx logo looks like figure \ref{fig:nginx1}, but if we want to reference this form another page.Nginx logo looks like figure \ref{fig:nginx1}, but if we want to reference this form another page.
Nginx logo looks like figure \ref{fig:nginx1}, but if we want to reference this form another page.Nginx logo looks like figure \ref{fig:nginx1}, but if we want to reference this form another page.
Nginx logo looks like figure \ref{fig:nginx1}, but if we want to reference this form another page.Nginx logo looks like figure \ref{fig:nginx1}, but if we want to reference this form another page.

\section{Introduction}

Hello world. This is the first section.
\section{Introduction}

Hello world. This is the first section.

Nginx logo looks like figure \ref{fig:nginx1}, but if we want to reference this form another page.Nginx logo looks like figure \ref{fig:nginx1}, but if we want to reference this form another page.
Nginx logo looks like figure \ref{fig:nginx1}, but if we want to reference this form another page.Nginx logo looks like figure \ref{fig:nginx1}, but if we want to reference this form another page.

\subsection{First Subsection}
Praesent imperdietmi nec ante. Donec ullamcorper, felis non sodales...

\subsubsection{First SubSubSection}

Nginx logo looks like figure \ref{fig:nginx1}, but if we want to reference this form another page.Nginx logo looks like figure \ref{fig:nginx1}, but if we want to reference this form another page.
Nginx logo looks like figure \ref{fig:nginx1}, but if we want to reference this form another page.Nginx logo looks like figure \ref{fig:nginx1}, but if we want to reference this form another page.

\section{Second Section}

Hello world. This is the second section.

Nginx logo looks like figure \ref{fig:nginx1}, but if we want to reference this form another page.Nginx logo looks like figure \ref{fig:nginx1}, but if we want to reference this form another page.
Nginx logo looks like figure \ref{fig:nginx1}, but if we want to reference this form another page.Nginx logo looks like figure \ref{fig:nginx1}, but if we want to reference this form another page.

Lorem ipsum dolor sit amet, consectetuer adipiscing elit.

\subsection{Second Subsection}

Praesent imperdietmi nec ante. Donec ullamcorper, felis non sodales...

Nginx logo looks like figure \ref{fig:nginx1}, but if we want to reference this form another page.Nginx logo looks like figure \ref{fig:nginx1}, but if we want to reference this form another page.
Nginx logo looks like figure \ref{fig:nginx1}, but if we want to reference this form another page.Nginx logo looks like figure \ref{fig:nginx1}, but if we want to reference this form another page.

\subsection{Second Subsection}
Praesent imperdietmi nec ante. Donec ullamcorper, felis non sodales...

\subsection{Third Subsection}
Praesent imperdietmi nec ante. Donec ullamcorper, felis non sodales...

\section*{Unnumbered Section}
Lorem ipsum dolor sit amet, consectetuer adipiscing elit.  

\begin{center} % this is how tables without lines are introduced
    \begin{tabular}{  c c }       
     cell1 & cell2 \\ 
     cell4 & cell5 \\  
     cell7 & cell8    
    \end{tabular}
\end{center}

\begin{table}[ht!]
    \begin{center} % this is how tables with lines are introduced
        \begin{tabular}{ |c|c| }
        \hline
        cell1 & cell2 \\ 
        \hline
        cell4 & cell5 \\
        \hline  
        cell7 & cell8 \\ 
        \hline 
        \end{tabular}
    \end{center}
    \caption{Table to test captions and labels}
    \label{table:data2}
\end{table}
        



Some of the \textbf{greatest} discoveries in were made by \textbf{\textit{accident}}.

Now we will enumerate this document index:


\begin{enumerate} % to create an enumerated list
    \item What's NGINX
    \item what we will use NGINX for
    \item introduction to nginx
\end{enumerate}

\begin{itemize} % to create an enumerated list
    \item What's NGINX
    \item what we will use NGINX for
    \item introduction to nginx
\end{itemize}

Did I told you $F=ma$

Oh, but check speed's ecuation
\begin{equation}
    E=m
\end{equation}

\end{document}
